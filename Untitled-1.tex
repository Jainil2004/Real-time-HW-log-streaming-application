\documentclass[a4paper,12pt]{report}
\usepackage{graphicx}
\usepackage{hyperref}
\usepackage{geometry}
\usepackage{amsmath, amssymb, amsthm}

\geometry{left=1.055118in, right=0.91in, top=1in, bottom=1in}

\begin{document}

\begin{titlepage}
    \centering
    \vspace*{1cm}
    \Huge
    \textbf{Title}
    \vfill
    \LARGE
    \textbf{\textit{A\\
    Project Report}}\\
    \textit{submitted in partial fulfillment of the\\
    requirements for the award of the degree of}\\
    \vfill
    \textbf{BACHELOR OF TECHNOLOGY}\\
    in\\
    \textbf{COMPUTER SCIENCE \& ENGINEERING}\\
    \vfill
    \textbf{by}\\
    \large
 \begin{tabular}{c c}
        \textbf{Name} & \textbf{Roll No.} \\ 
        XXXXX  & 1234567890 \\
        YYYY   & 1234567890 \\
        ZZZZ   & 1234567890 \\
        ABCD   & 1234567890 \\
\end{tabular}
    
    \vfill
    \textbf{\textit{Under the guidance of}}\\
    TTTTTTTT\\
    \vfill
    \includegraphics[width=0.4\textwidth]{UPES_logo.png}\\
    \vfill
    \textbf{School of Computer Science, UPES}\\
     Bidholi, Via Prem Nagar, Dehradun, Uttarakhand\\
    Month -- 2025
\end{titlepage}

\tableofcontents

\section{1. Introduction}

\subsection{1.1 Purpose}
The Real-Time Log Analysis System monitors and analyzes computer hardware logs in real-time, focusing on parameters such as temperature, voltage, and clock speed. It features a customizable log schema and fault tolerance mechanisms to ensure continuous operation. The system serves both commercial and personal users, enabling administrators to monitor large deployments and enthusiasts to optimize performance.

The system provides insights into hardware health and includes an alert mechanism for abnormal activities such as power failures, thermal throttling, and excessive core frequencies.

\subsection{1.2 Scope}
The system collects logs from HWInfo, processes them using Apache Kafka and Apache Spark, and stores them in Elasticsearch. Features include:
\begin{itemize}
    \item Automated monitoring of hardware logs.
    \item Anomaly detection using statistical and machine learning techniques.
    \item Dashboard visualization for real-time insights.
    \item Alert generation for critical events (e.g., overheating, abnormal voltage changes).
\end{itemize}

\subsection{1.3 Definitions, Acronyms, and Abbreviations}
\begin{itemize}
    \item Kafka: Distributed event streaming platform for log ingestion.
    \item Spark: Real-time data processing framework.
    \item Elasticsearch: Search and analytics engine for storing logs.
    \item HWInfo: Hardware monitoring tool generating log files.
    \item Anomaly Detection: Identifying unusual patterns in hardware behavior.
\end{itemize}

\subsection{1.4 References}
Research papers on real-time log analysis, anomaly detection, and big data tools.

\subsection{1.5 Overview}
This document outlines functional and non-functional requirements, external interfaces, and system design constraints.

\section{2. Overall Description}

\subsection{2.1 Product Perspective}
The system is a standalone application integrating big data technologies for efficient log processing, anomaly detection, and fault tolerance. It leverages Apache Kafka, Spark Streaming, and Elasticsearch to ensure rapid ingestion, processing, and storage.

\subsection{2.2 Product Functions}
\begin{itemize}
    \item Log Ingestion: HWInfo logs streamed into Kafka.
    \item Real-Time Processing: Spark detects anomalies.
    \item Data Storage and Querying: Logs stored in Elasticsearch.
    \item Dashboard and Alerts: Visualization and notifications.
\end{itemize}

\subsection{2.3 User Characteristics}
\begin{itemize}
    \item System Administrators: Require hardware health insights.
    \item Developers and Researchers: Need log analysis for diagnostics.
    \item Enthusiasts: Seek real-time monitoring.
\end{itemize}

\subsection{2.4 Constraints}
\begin{itemize}
    \item Must support high-velocity log ingestion.
    \item Must process logs in under 500ms.
    \item Must store logs efficiently.
\end{itemize}

\section{3. System Hardware Requirements}
\subsection{3.1 Recommended System Requirements}
\begin{itemize}
    \item CPU: x86\_64, Intel/AMD 4+ cores.
    \item RAM: 8GB+
    \item OS: Linux/Windows (WSL2)
    \item Software: Docker, Kafka, Spark, Elasticsearch.
\end{itemize}

\subsection{3.2 Minimum System Requirements}
\begin{itemize}
    \item CPU: x86\_64, Intel/AMD 2+ cores.
    \item RAM: 8GB
    \item OS: Linux/Windows (WSL2)
    \item Software: Docker, Kafka, Spark, Elasticsearch.
\end{itemize}

\section{4. Functional Requirements}
\subsection{4.1 Log Ingestion}
\subsection{4.2 Data Processing and Anomaly Detection}
\subsection{4.3 Data Storage and Retrieval}
\subsection{4.4 Dashboard and Notifications}

\section{5. External Interface Requirements}
\subsection{5.1 User Interfaces}
\subsection{5.2 Hardware Interfaces}
\subsection{5.3 Software Interfaces}

\section{6. Non-Functional Requirements}

\section{7. System Architecture}
\subsection{7.1 Architectural Diagram}
\subsection{7.2 Data Flow}

\end{document}
